\documentclass[letterpaper,11pt]{article}

\usepackage[utf8]{inputenc}
\usepackage{graphicx}
\usepackage{fourier}
\usepackage[spanish]{babel}
\usepackage{amsmath}
\usepackage{amssymb}
\usepackage{bm}
\usepackage{hyperref}

\title{Generalización del método LAMDA basado en escalas biplolares}
\author{Julio Waissman Vilanova}
\date{Universidad de Sonora \\
Departamento de Matemáticas\\
\href{mailto:juliowaissman@mat.uson.mx}{\texttt{juliowaissman@mat.uson.mx}}}

\begin{document}
\maketitle

\begin{abstract}
  Las escalas bipolares sirven para representar información en la cual
  el sentido de pertenencia es importante. Dichas escalas son muy
  utilizadas cuando se realizan comparaciones o calificaciones en
  forma de variables linguísticas. En este trabajo se propone una
  generalización del metodo LAMDA de aprendizaje supervisado y no
  supervisado considerando como piedra angular del método el hecho de
  representar conocimiento que naturalmente se representa en forma de
  escala bipolar. Este es un trabajo inicial para establecer las ideas
  básicas, las cuales pueden servir para desarollar una base de un
  sistema simple basado en el concepto de \emph{grado de
    adecuación}. Este resumen va a cambiar conforme se avance en el
  tema.
 
\end{abstract}

\section{Introducción}
\label{sec:introduccion}

\section{Escalas bipolares}
\label{sec:escalas-bipolares}

Primer hay que aclarar los diferentes tipo sde bipolaridad y que
significan cada uno, y dar particular énfasis al tipo de bipolaridad
que pensamos que debería de ser el que revisaramos.

\section{El método LAMDA}
\label{sec:el-metodo-lamda}

Aqui cabría hacer las explicaciones a partir de las ideas de base
expresadas tanto en la tesis de López de Matras como en los artículos
originales de Joseph. La idea es que quede claro que desde el
principio la idea de LAMDA fue el manejo de inforación bipolar, si
bien no se encontraba muy estudiada la idea en ese entonces.

\section{Generalización a escala bipolar}
\label{sec:gener-escala-bipol}

Aqui es necesario dejar bien claro en que sentido se realiza la
bipolaridad y porque vamos a imponer ciertas restricciones y formas de
hacer esto, de forma que quede bastante clara la manera en que lo
estamos haciendo.

\section{Aprendizaje Supervisado}
\label{sec:aprend-superv}

\section{Aprendizaje no supervisado}
\label{sec:aprend-no-superv}

\section{Conclusiones y trabajos futuros}
\label{sec:concl-y-trab}




\end{document}
